\section{Introduzione generale}

\subsection{Assegnazione del progetto}
Qui di seguito verrà esplicato il progetto che mi è stato assegnato per il superamento della parte di Laboratorio di Algoritmi e Strutture Dati.

\begin{itemize}
    \item Alberi Binari di Ricerca vs B-Alberi
\end{itemize}

\subsection{Breve descrizione dello svolgimento degli esercizi}
Per ogni esercizio suddivideremo la sua descrizione in 4 parti fondamentali:

\begin{itemize}
    \item \textbf{Spiegazione teorica del problema} : qui è dove si descrive il problema che andremo ad affrontare in modo teorico partendo dagli assunti del libro di Algoritmi e Strutture Dati e da altre fonti.
    \item \textbf{Documentazione del codice} : in questa parte spieghiamo come il codice dell'esercizio viene implementato
    \item \textbf{Descrizione degli esperimenti condotti} : partendo dal codice ed effettuando misurazioni varie cerchiamo di verificare le ipotesi teoriche
    \item \textbf{Analisi dei risultati sperimentali} : dopo aver svolto i vari esperimenti riflettiamo sui vari risultati ed esponiamo una tesi
\end{itemize}

\subsection{Specifiche della piattaforma di test}
La piattaforma di test sarà la stessa per ogni esercizio che vedremo. Partiamo dall'hardware del computer fondamentale da conoscere per questo esercizio:

\begin{itemize}
    \item \textbf{Modello} : Apple MacBookAir 8,1
    \item \textbf{CPU} : Intel 1,6 GHz Intel Core i5 dual-core
    \item \textbf{RAM} : 8 GB 2133 MHz LPDDR3
    \item \textbf{SSD} (interno) : APPLE SSD AP0256M 250,69GB
\end{itemize}

Il linguaggio di programmazione utilizzato sarà Python e la piattaforma in cui il codice è stato scritto e 'girato' è l'IDE \textbf{PyCharm Professional 2023.3.2}. La stesura di questo testo è avvenuta tramite l'utilizzo dell'editor online \textbf{Overleaf}.
